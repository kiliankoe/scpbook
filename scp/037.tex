\toclesssection{SCP 037 - Dwarf Star}
\addcontentsline{toc}{section}{SCP 037 - Dwarf Star}

\textbf{Item \#:} SCP-037

\textbf{dObject Class:} Safe

\textbf{Special Containment Procedure:} SCP-037 is magnetically contained at Site-32. It is housed in a small room composed of 1 m thick heat/radiation-proof polymer and evacuated of any atmosphere. Should the integrity of the enclosure become compromised, the emergency system will generate a low power argon plasma shield until it is repaired. Those repairing the walls should wear a climate-controlled suit provided by site personnel to avoid heat stroke. In the contingency of a [DATA EXPUNGED].

\textbf{Description:} The artifact was discovered in 19\censor{XX} above the Beaufort Sea at approximately the North Magnetic Pole. Intense electromagnetic interference was reported by Canadian Forces Station (CFS) Alert, followed by an extremely luminous object descending toward the ocean from the sky. The SCPS Guardian responded and discovered the object wavering in an erratic trajectory about 200 m above the surface of the water. Once containment procedures were devised it was transported to Site-32 for study.

SCP-037 is apparently a star exactly 5 cm (2 in) in diameter, with a luminosity of about 1*10$^{-12}$ times that of our sun and a surface temperature of about 5000 K (determined by UBVRI analysis). The age and origin of SCP-037 is currently unknown; however, its nuclear activity is being carefully monitored should it exhibit any dangerous irregularities. Spectral analysis and comparison to known celestial bodies suggest that it is a typical (other than its uncharacteristic size) star quickly undergoing the transition to a red giant, though it is unknown if established theories of star formation and aging apply. It is thought to have entered the Earth's magnetosphere via the North Magnetic Pole.

Containment and transport of SCP-037 have been achieved by the use of powerful electromagnets, to which the artifact aligns itself according to its own magnetic field. Over \censor{XX} years of study, the star has undergone a shift in emitted EM radiation, suggesting that it is undergoing stellar evolution at a vastly accelerated rate. If standard stellar models hold up, this will soon result in an increase in radius by a factor of 100 to 300 times. Emergency containment contingencies are being prepared for that case. Further progression of the star's life cycle will likely terminate in a stellar nova, which is estimated to have a yield of \censor{XXXXXXXXXXX}. Extrapolations predict this to occur in \censor{XXXXXXXXXXX}. Research is underway for a method to arrest this development.